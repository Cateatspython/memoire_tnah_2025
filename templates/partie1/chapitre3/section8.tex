\subsection{La création d'interfaces généreuses}


La notion de \og \textit{generous interface} \fg a été mentionnée pour la première fois par Mitchell Whitelaw lors d'un ICA Congress à Brisbane en Australie en 2013\footcite{GenerousInterfacesArchival}.

Il reprend ce terme dans un article publié en 2015 dans lequel il exhorte les institutions patrimoniales à arrêter l'usage de la barre de recherche\footcite{whitelawGenerousInterfacesDigital2015}. 

Il commence par exposer la métaphore de la galerie d'art. Cette dernière aurait le même fonctionnement d'une barre de recherche. Pour explorer les œuvres, nous serions contraints de formuler une requête à la personne à l'accueil. Il serait possible de répéter l'action autant de fois que nous voudrions mais toujours sous la forme d'une demande. Nous ne pourrions donc pas déambuler dans différentes salles pour découvrir les œuvres sans avoir une idée précise de ce que nous voulons voir. Cette métaphore est représentative de ce qu'il se passe sur beaucoup d'interfaces proposées par les institutions patrimoniales. C'est pour résoudre ce problème qu'il propose la solution de la \og \textit{generous interface} \fg.

Pour décrire le comportement de l'utilisateur sur ce genre d'interface, il reprend une autre métaphore, celle du \og \textit{flaneur d'informations} \fg de Marian Dörk\footcite{dorkInformationFlaneurFresh2011}. Ce dernier s'est inspiré de la figure du flâneur parisien du XIXe siècle qui se démarque par sa curiosité, sa créativité et son esprit critique. 

Pour Mitchell Whitelaw, les interfaces classiques avec une barre de recherche ne sont pas propices à ce type d'attitude. En effet, la complexité et l'ampleur des collections patrimoniales peuvent limiter l'utilisateur dans son exploration s'il ne connaît pas très bien le corpus. Ce problème concerne même les utilisateurs \og \textit{experts} \fg car ces derniers peuvent passer à côté d'œuvres cachées qui auraient pu faire l'objet d'une découverte\footcite{whitelawGenerousInterfacesDigital2015}. 

Il s'appuie alors sur deux concepts qu'il emprunte à d'autres chercheurs. 
Nous avons d'abord celui du \og \textit{visual information seeking} créé par Ben Shneiderman en 1996 pour définir la structure de l'interface parfaite\footcite{shneidermanEyesHaveIt}. Cette dernière doit se diviser au moins en trois parties : 

\begin{itemize}
	\item La \textit{vue d'ensemble} permet d'avoir une vision globale de la collection.
	\item Le \textit{zoom et filtrage} permet d'isoler un sous-ensemble.
	\item Pour finir, il est possible de voir les \textit{détails} qui sont en réalité les différents objets qui composent la collection.
\end{itemize}
A ces trois parties, s'ajoutent trois autres fonctionnalités à intégrer : 

\begin{itemize}
	\item Etablir des relations entre les différents objets afin de visualiser un contexte global ou des tendances.
	\item Conserver un historique de l'utilisateur pour ses prochaines visites sur l'interface.
	\item Donner à l'utilisateur la possibilité d'extraire les résultats pour ses recherches
\end{itemize}
Il est également possible d'emboîter plusieurs interfaces similaires à cette description pour obtenir des vues multiples et complémentaires entre elles
Le second concept emprunté par Mitchell Whitelaw est celui d' \og \textit{information surrogate}\fg créé par Stephan
Greene\footcite{greenePreviewsOverviewsDigital2000}. Il s'agit d'une autre approche qui consiste à utiliser un aperçu comme un substitut pour représenter un ensemble d'objet. Cela peut être une alternative à la vue d'ensemble de Ben Shneiderman.

Le concept de \og \textit{generous interface}\fg a reçu beaucoup d'approbation dans la communauté scientifique. Il est repris par George Oates lors d'une conférence EuropeanaTech en 2015\footcite{EuropeanaTech2015} à Paris pour implorer les institutions patrimoniales à laisser de côté le modèle de la barre de recherche pour leurs interfaces. EuropeanaTech Insigh lui consacre également trois articles en 2019\footcite{Issue11Generous}.


\subsection{La découverte par la sérendipité}

La terme de sérendipité est emprunté au mot anglais \og \textit{serendipity} \fg créé par Horace Walpole au XVIIIe siècle. Selon le \textit{Dictionnaire de l'Académie Française}, la sérendipité désigne un \og don de faire par hasard des découvertes fructueuses \fg et \og une forme de disponibilité intellectuelle qui permet de tirer de riches enseignements d'une trouvaille inopinée ou d'une erreur\fg\footcite{SerendipiteAcademieFrancaise}.

Réaliser une \textit{generous interface} avec des visualisations serait donc un moyen d'inciter, d'encourager et de récupérer l'information dérivée d'une découverte par sérendipité.

Pour amener à la sérendipité, il est intéressant de présenter son interface de manière structurée et soigner comme dans les collections d'une bibliothèque ou d'un musée. L'utilisateur est alors incité à se promener sur le site comme il pourrait le faire dans la vie réelle. Le principe de recommandation et l'affichage d'objet similaires à celui que nous recherchons sont deux idées amène à réaliser des découvertes fortuites\footcite{windhagerVisualizationCulturalHeritage2019}.


\subsection{Le dépassement des capacités humaines}


La visualisation peut nous permettre de distinguer des tendances et des irrégularités au sein du collection d'objet patrimoniaux que nous n'aurions pas remarquer si nous avions fait les recherches manuellement. Cependant, quand cette dernière rencontre des technologies poussées comme l'intelligence artificielle, elle peut aller plus loin encore.

Nous pouvons prendre pour exemple le projet \textit{Philherit} qui traite les problématiques liées à l'héritage en philosophie dans un large corpus grâce aux nouveaux outils numériques. En effet, au XIXe siècle, sa proximité avec la question de la justice sociale la rend centrale et de nombreux écrits variés voient le jour. 

Pour étudier ce corpus, les chercheurs du projet ont extraits sept bases de données issues de \textit{Gallica}. Par la suite, ils ont réparti les textes dans différents thèmes identifiés par le modèle de traitement automatique du langage naturel \gls{bert}. Ce dernier a ensuite généré automatiquement une visualisation sous la forme d'un nuage de mot qui reprend les grands thèmes. Plus un mot est présents dans le corpus, plus sa taille dans le nuage de mot sera conséquente. Cela permet de dégager les questions les plus abordées dans le corpus et d'accéder aux ouvrages qui les concernent\footcite{griveauVisualisationDonneesAu2025}.

Comme nous venons de le voir, l'utilisation de l'intelligence artificielle sur un corpus peut nous aider à générer des visualisations intéressantes pour aider les chercheurs et favoriser la découverte par sérendipité. 