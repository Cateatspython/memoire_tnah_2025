Les sources astronomiques constituent un exemple parfait pour l'étude des transmissions d'idées à travers les époques et les aires géographiques par le biais d'éléments iconographiques. En effet, chaque culture apporte des améliorations et des adaptations à ces derniers pour correspondre au mieux aux besoins spécifiques de leurs pratiques. L'observation des corps célestes ainsi que l'évolution des mathématiques ont contribué à l'essor de l'astronomie. 

Parmi elles, l'\almageste, l'un des textes les plus importants de l'astronomie antique écrit par Claude Ptolémée, témoigne de la circulation d'un système géocentrique du monde.

Claude Ptolémée est un astronome du \II siècle appartenant à l'école d'Alexandrie. Il serait intervenu dans le monde scientifique grec entre 90 et 168 ap. J.-C. à un moment où l'astronomie a peu évolué depuis les travaux d'Hipparque entre 147 et 127 av. J.-C. Il est connu pour le modèle géocentrique de l'univers qui porte son nom. Il développe cette théorie dans son ouvrage le plus connu, la \textit{Grande Syntaxe mathématique}, désignée le plus souvent sous le nom d'\almageste\footcite{verdetLaubeLastronomieLaurore1990}. 


Cette œuvre contient un catalogue des étoiles, un traité complet de trigonométrie plane et sphérique, une liste des instruments essentiels à avoir dans un observatoire et une partie consacrée aux mouvements des astres.
C'est dans cette dernière partie, qu'il reprend un modèle géocentrique déjà connu pour le perfectionner en précisant la théorie des épicycles et en ajoutant l'innovation mathématique du point équant\footcite{AstronomyAncientGreece2025}. 
Dans un modèle géocentrique, la Terre est au centre de l'univers et les différents astres, c'est-à-dire le Soleil, la Lune et les planètes, gravitent autour d'elle selon des cercles dits épicycles et déférents\footnote{L'épicycle est le cercle sur lequel circulent les planètes. Le centre de ce cercle bouge sur un cercle coplanaire nommé déférent qui est centré sur la Terre\footcite{EpicyclesPtolemee}.}\footcite{costabelCLAUDEPTOLEMEE90}.
Les observations et la description des procédures mathématiques appliquées pour définir les paramètres du modèle de Ptolémée forment une grande partie de l'\almageste\footcite{AstronomyAncientGreece2025}. 

La diffusion de l'\textit{Almageste} est remarquable. Elle s'étend sur plus de 1400 ans et concerne à la fois le bassin méditerranéen, le monde arabe, l'Occident latin mais aussi certaines régions d'Asie. 
Nous devons la transmission de cette œuvre de l'Antiquité à la Renaissance à de nombreux copistes, traducteurs et commentateurs. 

Après la mort de Ptolémée, l'astronomie grecque connaît peu d'innovation. Les travaux réalisés par ses successeurs sont principalement des commentaires de ses écrits. Les savants ne cherchent pas à remettre en question son modèle qui demeure la norme durant de nombreux siècles et dans de nombreuses aires géographiques. Cependant, il existe tout de même quelques variantes avec l'apparition de différentes traditions de l'œuvre.

Les diagrammes, élément central dans les sources astronomiques, joue un rôle important dans l'étude de la diffusion de l'œuvre de Ptolémée. Il s'agit de représentations iconographiques utilisées en astronomie pour illustrer les positions, les mouvements et les relations entre les corps célestes.

% Lire Hamburger (Jeffrey F.), Roxburgh (David J.) et Safran (Linda), The Diagram as Paradigm : Cross-Cultural Approaches, Cambridge, MA, 2022 (Dumbarton Oaks Byzantine Symposia and Colloquia). --> différences traditions diagrammes pour le monde occidental, arabe et byzantin
Pour la tradition grecque de l'\almageste, deux savants réalisent des commentaires de l'œuvre peu de temps après la disparition de Ptolémée.
Théon d'Alexandrie qui aurait vécu autour de 364 ap. J.-C. est l'un des commentateurs les plus prolifiques des travaux de Ptolémée. Cependant, son commentaire de l'\almageste n'a pas entièrement survécu aux différentes époques et certaines parties sont aujourd'hui manquantes\footcite{tihonLivreRetrouveCommentaire1987}. Vers 340 ap. J.-C., un deuxième savant grec, le mathématicien Pappus d'Alexandrie rédige un commentaire de l'\textit{Almageste} dans le \textit{Livre IV} de ses \textit{Collections mathématiques}. Il s'agit d'un ouvrage dans lequel il expose de manière complète et systématique toutes les connaissances de son époque en apportant des explications et des approfondissements\footcite{meyerPAPPUS1999}. 

Par la suite, l'œuvre se diffuse hors du monde hellénistique vers le monde arabe. Le savant persan Mohammad Nasir al-Din al-Tūs rédige un \textit{Tahrir al-Majiṣtī}, ce qui signifie \textit{Commentaire sur l'Almageste}\footcite{universalisMOHAMMADNASIRALDIN2008}. 


Averroès est aussi à l'origine d'un \textit{Abrégé de l'Almageste} écrit en 1159-1162. La particularité de cette œuvre vient du fait qu'elle ait été écrite en Andalousie au XIIe siècle dans un contexte de remise en question de l'astronomie ptolémaïque. En effet, des savants comme Maïmonide, Ibn Bajja ou Ibn Tufayl pensaient qu'il était nécessaire de réformer les idées de Ptolémée pour qu'elles concordent parfaitement avec la vision aristotélicienne du monde. Averroès fait parti de ce mouvement mais lorsqu'il rédige l'\textit{Abrégé de l'Almageste}, il n'a pas la volonté de corriger l'œuvre. En effet, il préfère attendre une réforme totale de l'astronomie ptolémaïque et se résigne donc à suivre les idées de son temps sur l'\textit{Almageste}.
Quelques petites erreurs involontaires se sont sans doute glissées dans son \textit{Abrégé de l'Almageste}. Même s'il possède un solide bagage scientifique, Averroès n'est pas astronome à l'origine\footcite{layAverroesAbregeDastronomie1998}. Nous ne possédons qu'une version de cette œuvre traduite en hébreu par Jacob Anatoli dans le cadre du mécénat de l'empereur Frédéric II au XIIIe siècle. Il n'existe plus aucun témoin de la version arabe et il semblerait que cette dernière n'ait jamais été traduite en latin\footcite{layAverroesHebraicusInedit2005}.

Plusieurs siècles plus tard, Gérard de Crémone importe l'œuvre de Ptolémée dans l'Occident chrétien. 
Ce dernier se rend à Tolède dans le but d'avoir accès aux traductions arabes de l'\textit{Almageste} alors que l'œuvre n'existe pas encore en latin. Grâce à sa connaissance de l'arabe et à ses solides connaissances en logique, mathématique et astronomie, Gérard de Crémone réalise la première traduction latine de l' \textit{Almageste}. Par la suite, il n'hésite pas à effectuer plusieurs révisions de la traduction en consultant de nouveaux manuscrits en arabe. Il existe donc plusieurs états du texte traduit par Gérard de Crémone. Paul Kunitzsch, un chercheur allemand en études arabes, a relevé trois états de sa traduction. Il a examiné trente-quatres témoins de l'œuvre mais seulement quatre d'entre-elles possède explicitement le nom de Gérard de Crémone. Il semblerait que des copies de son travail soient sorties progressivement de Tolède au fur et à mesure qu'il traduisait l'œuvre et qu'il apportait des corrections\footcite{jacquartTraductionsAuFil2018}.

La \gls{bnf} possède plusieurs copies de la traduction de l'\textit{Almageste} par Gérard de Crémone. Elle possède notamment la plus ancienne copie de son premier état, le manuscrit \textit{Latin 14738}\footcite{jacquartTraductionsAuFil2018}. Elle détient également une copie latine réalisée en 1213 à Paris avec des annotations\footcite{ptolemaeusPtolomeusAlmagestumTransl1213}, le manuscrit \textit{Latin 16200}. 
Dans l'une des initiales historiées du manuscrit, nous pouvons observer une illustration en hommage  à Ptolémée le représentant sous la forme d'un roi trônant en majesté\footcite{TraductionLatineLAlmageste}. 

Ces différentes copies, traductions et commentaires montrent à quel point la diffusion de l'œuvre est étendue. La vision géocentrique de l'univers de Ptolémée n'est remise en question qu'au XVIe siècle suite aux travaux de Nicolas Copernic. Dans son \textit{De Revolutionibus orbium caelestium}, il défend la théorie de l'héliocentrisme dans laquelle le Soleil se trouve au centre du monde\footcite{verdetHELIOCENTRISME2008}. \\

