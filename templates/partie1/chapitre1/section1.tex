Les sources scientifiques constituent un exemple parfait pour l'étude des transmissions d'idées à travers les époques et les aires géographiques.
En effet, elles véhiculent des savoirs qui sont partagés, repris et corrigés dans le but de s'approcher au plus de la vérité.
Parmi elles, les sources relatives à l'astronomie, comme le corpus ptolémaïque, témoignent de la circulation de concepts et de représentations du monde qui traversent les siècles et les cultures.

\subsection{Le corpus ptolémaïque}
Claude Ptolémée est un astronome appartenant à l'école d'Alexandrie dont nous ne savons presque rien de sa vie personnelle.
Il serait né vers 90 de notre ère et mort vers 168.
Il est considéré comme le dernier grand astronome grec de son époque à un moment où l'astronomie a peu évolué depuis Hipparque, un savant ayant vécu entre 147 et 127 avant notre ère.
Il doit sa renommée au modèle ptolémaïque dont il est le théoricien.
Il s'agit d'un système géocentrique de l'univers qui place la Terre au centre du monde.
Il développe cette théorie dans son ouvrage le plus connu, la \textit{Grande Syntaxe mathématique}, désignée le plus souvent sous le nom d'\textit{Almageste}\footcite{verdetLaubeLastronomieLaurore1990}.
Cette œuvre contient un catalogue des étoiles, un traité complet de trigonométrie plane et sphérique, une liste des instruments essentiels à avoir dans un observatoire et une partie consacrée aux mouvements des astres.
C'est dans cette dernière qu'il expose sa thèse selon laquelle l'univers est organisé en un modèle géocentrique.
Ptolémée considère que la Terre est au centre de l'univers et que les différents astres, c'est-à-dire le Soleil, la Lune et les planètes, gravitent autour d'elle selon des cercles dits épicycles et déférents\footcite{costabelCLAUDEPTOLEMEE90}.
Il schématise ses pensées à l'aide d'éléments iconographiques~: les diagrammes.

\subsection{La diffusion de l'œuvre de Ptolémée}
La diffusion des diagrammes de tradition ptolémaïque présents dans l'\textit{Almageste} est considérable.
Elle s'étend sur plus de 1400 ans et concerne à la fois le bassin méditerranéen, le monde arabo-islamique, l'Occident chrétien mais aussi certaines régions d'Asie.

Cette transmission se manifeste déjà à travers l'étymologie du titre \og Almageste \fg.
En effet, à l'origine appelé \og H' math'matik' syntaxis \fg signifiant la \og Grande Syntaxe mathématique \fg,
il est transformé en un terme hybride entre l'arabe et le grec se traduisant par \og le plus grand \fg avant d'être latinisé en \og Almagestum \fg\footcite{raymondjonesPtolemyAccomplishmentsBiography2025}.
Cette évolution témoigne de la diffusion de l'œuvre dans les mondes grec et arabe puis plus tard dans l'Occident latin.
Nous devons la diffusion de cette œuvre de l'Antiquité à la Renaissance à de nombreux copistes, traducteurs et commentateurs.

Après la mort de son auteur, deux savants grecs l'ont commenté.
Théon d'Alexandrie qui aurait vécu autour de 364 est l'un des commentateurs de Ptolémée les plus prolifiques.
Malheureusement son commentaire de l'\textit{Almageste} n'a pas entièrement survécu aux différentes époques et certaines parties sont aujourd'hui manquantes.
Par exemple, le Livre III a presque totalement disparu de la tradition manuscrite et il est souvent remplacé par une réécriture de Nicolas Cabasilas datant du XIVe siècle.
En 1953, le philologue Chanoine Adolphe Rome retrouve une édition conservée dans un unique manuscrit le \textit{Laurentianus gr. 28/18}.
Néanmoins, celui-ci s'avère lacunaire~: Il ne reste que des fragments du Livre V et le Livre XI a entièrement disparu\footcite{tihonLivreRetrouveCommentaire1987}.
Vers 340, un deuxième savant grec, le mathématicien Pappus d'Alexandrie rédige un commentaire de l'\textit{Almageste} dans le Livre IV de ses \textit{Collections mathématiques},
ouvrage dans lequel il expose de manière complète et systématique toutes les connaissances de son époque en apportant des explications et des approfondissements\footcite{meyerPAPPUS1999}.

Plusieurs siècles plus tard, Gérard de Crémone importe l'œuvre de Ptolémée dans l'Occident chrétien.

Il réalise une traduction latine de l'\textit{Almageste} en 1213 à Paris avec des annotations\footcite{ptolemaeusPtolomeusAlmagestumTransl1213}.

Dans l'une des initiales historiées du manuscrit, nous pouvons observer une illustration en hommage à Ptolémée le représentant sous la forme d'un roi trônant en majesté\footcite{TraductionLatineLAlmageste}.

Dans le monde arabo-islamique, le savant persan Mohammad Nasir al-Din al-Tūs rédige un \textit{Tahrir al-Majiṣtī}, ce qui signifie \textit{Commentaire sur l'Almageste}\footcite{universalisMOHAMMADNASIRALDIN2008}.
Le célèbre Averroès est aussi à l'origine d'un \textit{Abrégé de l'Almageste} écrit en 1159-1162.
La particularité de cette œuvre vient du fait qu'elle ait été écrite en Andalousie au XIIe siècle dans un contexte de remise en question de l'astronomie ptolémaïque.
En effet, des savants comme Maïmonide, Ibn Bajja ou Ibn Tufayl pensaient qu'il était nécessaire de réformer les idées de Ptolémée pour qu'elles concordent parfaitement avec la vision aristotélicienne du monde.
Averroès faisait parti de ce mouvement mais lorsqu'il rédige l'\textit{Abrégé de l'Almageste}, il n'a pas la volonté de corriger l'œuvre.
En effet, il préfère attendre une réforme totale de l'astronomie ptolémaïque et se résigne donc à suivre les idées originelles de l'\textit{Almageste}.
Quelques petites erreurs involontaires se sont sans doute glissées dans son \textit{Abrégé de l'Almageste}.
Même s'il possède un solide bagage scientifique, Averroès n'est pas astronome à l'origine\footcite{layAverroesAbregeDastronomie1998}.
Nous ne possédons qu'une version de cette œuvre traduite en hébreu par Jacob Anatoli dans le cadre du mécénat de l'empereur Frédéric II au XIIIe siècle.
Il n'existe plus aucun témoin de la version arabe et il semblerait que cette dernière n'est jamais été traduite en latin\footcite{layAverroesHebraicusInedit2005}.

Ces différentes copies, traductions et commentaires montrent à quel point la diffusion de l'œuvre est étendue.
La vision géocentrique de l'univers de Ptolémée ne sera remise en question qu'au XVIe siècle suite aux travaux de Copernic.
Dans son \textit{De Revolutionibus orbium caelestium}, il défend la thèse de l'héliocentrisme dans laquelle le Soleil se trouve au centre du monde\footcite{verdetHELIOCENTRISME2008}.