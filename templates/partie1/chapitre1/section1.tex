\som{La mention d'EIDA devrait plutôt venir après la mention des sources astronomiques, sinon ça vient un peu comme un cheveu sur la soupe.}

Les sources astronomiques constituent un exemple parfait pour l'étude des transmissions d'idées à travers les époques et les aires géographiques. En effet, elles véhiculent des savoirs qui sont partagés, repris et corrigés dans le but de s'approcher au plus de la vérité. 

\som{Parler de Ptolémée lui-même avant de dire "corpus ptolémaïque" car cela fait surtout référence à la dynastie égyptienne des Ptolémée, d'autant plus lorsqu'on parle de l'Antiquité. Il vaut mieux parler directement de l'Almageste.}

Parmi elles, le corpus ptolémaïque, l'un des plus importants de l'Antiquité, témoigne de la circulation d'une conception du monde sous la forme d'un système géocentrique. 

\subsection{Le corpus ptolémaïque}
\som{On parle de "modèle ptolémaïque" ou "système ptolémaïque" plutôt que de corpus, ce n'est pas la même chose et on parle là d'un système mathématique et non d'un ensemble d'oeuvre donc c'est plus approprié. Dans la partie, se focaliser sur Ptolémée mathématicien plutôt que Ptolémée innovateur. Par ailleurs, il n'invente pas le géocentrisme, bien au contraire, il intègre une innovation mathématique (le point équant) qui affine le modèle géocentrique déjà établi et précise le calcul de trajectoire des corps célestes.}

Claude Ptolémée est un astronome appartenant à l'école d'Alexandrie \som{dont nous ne
savons presque rien de sa vie personnelle ; à supprimer, c'est vrai pour la plupart des auteurs de l'Antiquité mais l'intérêt historique de Ptolémée réside dans ses textes dont il reste beaucoup de témoins}. Il serait né vers 90 de notre ère et mort vers 168. Il est considéré comme le dernier grand astronome grec de son époque \som{pas sûr de cette info, peut-être se contenter de dire qu'il intervient à ce moment-là} à un moment où l'astronomie a peu évolué depuis Hipparque, un savant ayant vécu entre 147 et 127 avant notre ère \som{avant J.-C.}. Il doit sa renommée \som{idem faire attention avec ce type de formulation} au modèle ptolémaïque dont il est le théoricien. Il s'agit d'un système géocentrique de l'univers qui place la Terre au centre du monde. Il développe cette théorie dans son ouvrage le plus connu, la \textit{Grande Syntaxe mathématique}, désignée le plus souvent sous le nom d'\textit{Almageste}\footcite{verdetLaubeLastronomieLaurore1990}. 

Cette œuvre contient un catalogue des étoiles, un traité complet de trigonométrie plane et sphérique, une liste des instruments essentiels à avoir dans un observatoire et une partie consacrée aux mouvements des astres. C'est dans cette dernière qu'il expose sa thèse selon laquelle l'univers est organisé en un modèle géocentrique. \som{Attention, ce n'est pas "sa" thèse !} Ptolémée considère que la Terre est au centre de l'univers et que les différents astres, c'est-à-dire le Soleil, la Lune et les planètes, gravitent autour d'elle selon des cercles dits épicycles et déférents\footcite{costabelCLAUDEPTOLEMEE90} \som{Vocabulaire spécifique donc définition en nbp}. 

\subsection{La diffusion de l'œuvre de Ptolémée}
La diffusion de l'\textit{Almageste} est remarquable. Elle s'étend sur plus de 1400 ans et concerne à la fois le bassin méditerranéen, le monde arabo-islamique, l'Occident chrétien mais aussi certaines régions d'Asie. 

Cette transmission se manifeste déjà à travers l'étymologie du titre \og Almageste \fg. En effet, à l'origine appelé \og H' math'matik' syntaxis \fg signifiant la \og Grande Syntaxe mathématique \fg, il est transformé en un terme hybride entre l'arabe et le grec se traduisant par \og le plus grand \fg avant d'être latinisé en \og Almagestum \fg\footcite{raymondjonesPtolemyAccomplishmentsBiography2025}. Cette évolution témoigne de la diffusion de l'œuvre dans les mondes grec et arabe puis plus tard en Occident. Nous devons la diffusion de cette œuvre de l'Antiquité à la Renaissance à de nombreux copistes, traducteurs et commentateurs.  \som{Pas la peine de revenir sur l'évolution du titre selon moi}

\subsubsection{La diffusion dans le monde grec antique}
Après la mort de son auteur, deux savants grecs ont commenté l'\textit{Almageste}. 

Théon d'Alexandrie qui aurait vécu autour de 364 de notre ère est l'un des commentateurs les plus prolifiques des travaux de Ptolémée. Malheureusement \som{Pas de jugement de valeur dans un mémoire} son commentaire de l'\textit{Almageste} n'a pas entièrement survécu aux différentes époques et certaines parties sont aujourd'hui manquantes. Par exemple, le \textit{Livre III} a presque totalement disparu de la tradition manuscrite et il est souvent remplacé par une réécriture de Nicolas Cabasilas datant du XIVe siècle \som{Source ?}. En 1953, le philologue Chanoine Adolphe Rome a retrouvé une édition conservée dans le manuscrit \textit{Laurentianus gr. 28/18}. Néanmoins, celui-ci s'avère lacunaire : Il ne reste que des fragments du \textit{Livre V} et le \textit{Livre XI} a entièrement disparu\footcite{tihonLivreRetrouveCommentaire1987}. 

Vers 340 de notre ère, un deuxième savant grec, le mathématicien Pappus d'Alexandrie rédige un commentaire de l'\textit{Almageste} dans le \textit{Livre IV} de ses \textit{Collections mathématiques}, ouvrage dans lequel il expose de manière complète et systématique toutes les connaissances de son époque en apportant des explications et des approfondissements\footcite{meyerPAPPUS1999}.

\som{Intéressant de mentionner les deux commentateurs de Ptolémée, mais la partie devrait être tournée plus clairement sur la diffusion : je pense que ça peut tenir en une phrase et ne justifie pas une sous-partie entière.} 

\subsubsection{La diffusion dans le monde arabo-islamique}
\som{Monde arabe, on ne dit pas arabo-islamique. Arabo-musulman à la limite}
Dans le monde arabo-islamique, le savant persan Mohammad Nasir al-Din al-Tūs rédige un \textit{Tahrir al-Majiṣtī}, ce qui signifie \textit{Commentaire sur l'Almageste}\footcite{universalisMOHAMMADNASIRALDIN2008}. 
Le célèbre \som{idem on évite ce type de formulation} Averroès est aussi à l'origine d'un \textit{Abrégé de l'Almageste} écrit en 1159-1162. La particularité de cette œuvre vient du fait qu'elle ait été écrite en Andalousie au XIIe siècle dans un contexte de remise en question de l'astronomie ptolémaïque. En effet, des savants comme Maïmonide, Ibn Bajja ou Ibn Tufayl pensaient qu'il était nécessaire de réformer les idées de Ptolémée pour qu'elles concordent parfaitement avec la vision aristotélicienne du monde. Averroès fait parti de ce mouvement mais lorsqu'il rédige l'\textit{Abrégé de l'Almageste}, il n'a pas la volonté de corriger l'œuvre. En effet, il préfère attendre une réforme totale de l'astronomie ptolémaïque et se résigne donc à suivre les idées de son temps sur l'\textit{Almageste}.
Quelques petites erreurs involontaires se sont sans doute glissées dans son \textit{Abrégé de l'Almageste}. Même s'il possède un solide bagage scientifique, Averroès n'est pas astronome à l'origine\footcite{layAverroesAbregeDastronomie1998}. Nous ne possédons qu'une version de cette œuvre traduite en hébreu par Jacob Anatoli dans le cadre du mécénat de l'empereur Frédéric II au XIIIe siècle. Il n'existe plus aucun témoin de la version arabe et il semblerait que cette dernière n'ait jamais été traduite en latin\footcite{layAverroesHebraicusInedit2005}.

\som{De même, se focaliser sur la diffusion des idées plutôt que sur la vie de mathématiciens, qui ne permet pas d'expliquer pourquoi on s'intéresse à ce corpus et pourquoi les diagrammes spécifiquement sont intéressants.}

\subsubsection{La diffusion dans l'Occident chrétien}

\som{Monde latin plutôt qu'Occident chrétien, on ne l'utilise pas dans le projet et la question de la religion n'est pas centrale pour nous.}

Plusieurs siècles plus tard, Gérard de Crémone importe l'œuvre de Ptolémée dans l'Occident chrétien. 
Ce dernier se rend à Tolède dans le but d'avoir accès aux traductions arabes de l'\textit{Almageste} alors que l'œuvre n'existe pas encore en latin. Grâce à sa connaissance de l'arabe et à ses solides connaissances en logique, mathématique et astronomie \som{idem formulation trop personnelle}, Gérard de Crémone réalise la première traduction latine de l' \textit{Almageste}. Par la suite, il n'hésite pas à effectuer plusieurs révisions de la traduction en consultant de nouveaux manuscrits en arabe. Il existe donc plusieurs états du texte traduit par Gérard de Crémone. Paul Kunitzsch, un chercheur allemand en études arabes, a relevé trois états de sa traduction. Il a examiné trente-quatres témoins de l'œuvre mais seulement quatre d'entre-elles possède explicitement le nom de Gérard de Crémone. Il semblerait que des copies de son travail soient sorties progressivement de Tolède au fur et à mesure qu'il traduisait l'œuvre et qu'il apportait des corrections\footcite{jacquartTraductionsAuFil2018}.

La \gls{bnf} possède plusieurs copies de la traduction de l'\textit{Almageste} par Gérard de Crémone. Elle possède notamment la plus ancienne copie de son premier état, le manuscrit \textit{Latin 14738}\footcite{jacquartTraductionsAuFil2018}. Elle détient également une copie latine réalisée en 1213 à Paris avec des annotations\footcite{ptolemaeusPtolomeusAlmagestumTransl1213}, le manuscrit \textit{Latin 16200}. 
Dans l'une des initiales historiées du manuscrit, nous pouvons observer une illustration en hommage  à Ptolémée le représentant sous la forme d'un roi trônant en majesté\footcite{TraductionLatineLAlmageste}. 

\som{Idem pas la peine de détailler autant}

Ces différentes copies, traductions et commentaires montrent à quel point la diffusion de l'œuvre est étendue. La vision géocentrique de l'univers de Ptolémée n'est remise en question qu'au XVIe \som{Petites capitales pour les numéros de siècles} siècle suite aux travaux de Nicolas Copernic. Dans son \textit{De Revolutionibus orbium caelestium}, il défend la théorie de l'héliocentrisme dans laquelle le Soleil se trouve au centre du monde\footcite{verdetHELIOCENTRISME2008}. \\

Au sein du corpus ptolémaïque, un élément iconographique revient de manière récurrente : il s'agit du diagramme. Comme nous allons le voir, l'étude de ce dernier est un excellent moyen de retracer la diffusion d'une œuvre dans temps et dans le monde. 

\som{Ça devrait arriver beaucoup plus tôt et être le fil conducteur de la partie selon moi, la question de la diffusion en elle-même peut être explorée en une sous-partie unique pour être mieux intégrée ensuite à la partie sur les diagrammes.}