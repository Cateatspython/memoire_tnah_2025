\documentclass[a4paper,12pt,twoside]{book}
\usepackage[T1]{fontenc}
\usepackage{inputenc}
\usepackage{fontspec}
\usepackage{lmodern}
\usepackage[english,french]{babel}
\usepackage{xspace} % pour la gestion des espaces après les commandes
\usepackage{xcolor}
\usepackage{minted} % colored source code
\usepackage{csquotes}

% Mise en page École des chartes
\usepackage[margin=2.5cm]{geometry} % marges
\usepackage{setspace}
\onehalfspacing % interligne de 1.5
\setlength\parindent{1cm}

\usepackage{tocbibind}
\usepackage[backend=biber, sorting=nyt, style=enc, minbibnames=10, maxbibnames=10]{biblatex}
\addbibresource{bibliographie/biblio_astro.bib}
\nocite{*}
\defbibnote{intro}{Cette bibliographie présente toutes les ressources utilisées, de tout type, citées ou non, par simple ordre alphabétique.}

\usepackage[pdfusetitle, pdfsubject={Mémoire TNAH — Titre}, pdfkeywords={mot1, mot2, mot3}]{hyperref}

\usepackage{graphicx}
\usepackage{subcaption}

\author{Prénom Nom – M2 TNAH — ENC}
\title{Titre mémoire}

% ACRONYMS
\usepackage[automake, acronym, toc]{glossaries}
\makeglossaries
\setacronymstyle{short-long}
\newacronym{fair}{\textsc{fair}}{\emph{Findable Accessible Interoperable Reusable}}
\newacronym{api}{\textsc{api}}{\emph{Application Programming Interface}}
\newacronym{eida}{EIDA}{Editing and analysing hIstorical astronomical Diagrams with Artificial intelligence}

% COMMANDS
\newcommand{\enc}{École nationale des chartes\xspace}
\newcommand{\fair}{\gls{fair}\xspace}
\newcommand{\api}{\gls{api}\xspace}
\newcommand{\XIV}{\textsc{xiv}\ieme{}\xspace}
% Pour retirer le titre courant d'une page vide avant un chapitre
\newcommand{\clearemptydoublepage}{\newpage{\pagestyle{empty}\cleardoublepage}}
% Pour des sections non numérotées dans la table des matière
\newcommand\chapterNo[1]{
	\chapter*{#1}
	\markright{\MakeUppercase{#1}}
}

\definecolor{commenttext}{RGB}{139,69,19}
\definecolor{commentnote}{RGB}{201,87,6}

\newcommand{\sego}[2]{%
    \textcolor{commenttext}{\textit{#1}}%
    \textcolor{commentnote}{\textbf{[#2]}}%
}

\begin{document}
	\onehalfspacing
	\frontmatter

	\input{templates/page-titre}

	\thispagestyle{empty}
	\cleardoublepage

	\include{templates/resume}

	\chapterNo{Remerciements}
	\addcontentsline{toc}{chapter}{Remerciements}

	\chapterNo{Introduction}
	\addcontentsline{toc}{chapter}{Introduction}

	La mission principale de ce stage était de concevoir des preuves de concept pour des visualisations à partir des données des projets EIDA et VHS. Ces dernières ont pour but de tester la faisabilité et la pertinence de visualisations à intégrer au sein de l’application de traitement semi-automatique de données AIKON. Avant de les développer, j’ai rédigé un cahier des charges et un benchmark des solutions techniques envisageables. Ces livrables techniques m’ont permis de définir des objectifs clairs ainsi que les limites de mon projet. Afin de bien comprendre le fonctionnement de l’application AIKON et l’expérience utilisateur qu’elle procure, j’ai rédigé une documentation pour les différentes fonctionnalités de l’interface utilisateur. Pour clore ce stage, un Digital Humanities Seminar a été organisé pour partager aux chercheurs mes recherches, les différentes étapes de réalisation des visualisations et les résultats obtenus. Plus largement, ce travail m’a amené à réaliser une réflexion approfondie autour du concept de visualisation des données en sciences humaines et sociales que je vais exposer dans ce mémoire.

	\thispagestyle{empty}
	\cleardoublepage

	\mainmatter

	\part{Enjeux scientifiques et techniques de l'analyse automatisée des transmissions iconographiques}
	\chapter[Les diagrammes astronomiques]{Les diagrammes astronomiques comme objets d'étude des circulations visuelles}

	Le projet \gls{eida} a choisi de se concentrer sur l'étude des diagrammes astronomiques pour retracer les trajectoires d'idées relatives à l'histoire de l'astronomie dans le temps et dans l'espace.

	\section[Un corpus idéal pour l'étude des transmissions]{Les sources astronomiques~: un corpus idéal pour l'étude des transmissions}
	Les sources astronomiques constituent un exemple parfait pour l'étude des transmissions d'idées à travers les époques et les aires géographiques par le biais d'éléments iconographiques. En effet, chaque culture apporte des améliorations et des adaptations à ces derniers pour correspondre au mieux aux besoins spécifiques de leurs pratiques. L'observation des corps célestes ainsi que l'évolution des mathématiques ont contribué à l'essor de l'astronomie. 

Parmi elles, l'\almageste, l'un des textes les plus importants de l'astronomie antique écrit par Claude Ptolémée, témoigne de la circulation d'un système géocentrique du monde.

Claude Ptolémée est un astronome du \II siècle appartenant à l'école d'Alexandrie. Il serait intervenu dans le monde scientifique grec entre 90 et 168 ap. J.-C. à un moment où l'astronomie a peu évolué depuis les travaux d'Hipparque entre 147 et 127 av. J.-C. Il est connu pour le modèle géocentrique de l'univers qui porte son nom. Il développe cette théorie dans son ouvrage le plus connu, la \textit{Grande Syntaxe mathématique}, désignée le plus souvent sous le nom d'\almageste\footcite{verdetLaubeLastronomieLaurore1990}. 


Cette œuvre contient un catalogue des étoiles, un traité complet de trigonométrie plane et sphérique, une liste des instruments essentiels à avoir dans un observatoire et une partie consacrée aux mouvements des astres.
C'est dans cette dernière partie, qu'il reprend un modèle géocentrique déjà connu pour le perfectionner en précisant la théorie des épicycles et en ajoutant l'innovation mathématique du point équant\footcite{AstronomyAncientGreece2025}. 
Dans un modèle géocentrique, la Terre est au centre de l'univers et les différents astres, c'est-à-dire le Soleil, la Lune et les planètes, gravitent autour d'elle selon des cercles dits épicycles et déférents\footnote{L'épicycle est le cercle sur lequel circulent les planètes. Le centre de ce cercle bouge sur un cercle coplanaire nommé déférent qui est centré sur la Terre\footcite{EpicyclesPtolemee}.}\footcite{costabelCLAUDEPTOLEMEE90}.
Les observations et la description des procédures mathématiques appliquées pour définir les paramètres du modèle de Ptolémée forment une grande partie de l'\almageste\footcite{AstronomyAncientGreece2025}. 

La diffusion de l'\textit{Almageste} est remarquable. Elle s'étend sur plus de 1400 ans et concerne à la fois le bassin méditerranéen, le monde arabe, l'Occident latin mais aussi certaines régions d'Asie. 
Nous devons la transmission de cette œuvre de l'Antiquité à la Renaissance à de nombreux copistes, traducteurs et commentateurs. 

Après la mort de Ptolémée, l'astronomie grecque connaît peu d'innovation. Les travaux réalisés par ses successeurs sont principalement des commentaires de ses écrits. Les savants ne cherchent pas à remettre en question son modèle qui demeure la norme durant de nombreux siècles et dans de nombreuses aires géographiques. Cependant, il existe tout de même quelques variantes avec l'apparition de différentes traditions de l'œuvre.

Les diagrammes, élément central dans les sources astronomiques, joue un rôle important dans l'étude de la diffusion de l'œuvre de Ptolémée. Il s'agit de représentations iconographiques utilisées en astronomie pour illustrer les positions, les mouvements et les relations entre les corps célestes.

% Lire Hamburger (Jeffrey F.), Roxburgh (David J.) et Safran (Linda), The Diagram as Paradigm : Cross-Cultural Approaches, Cambridge, MA, 2022 (Dumbarton Oaks Byzantine Symposia and Colloquia). --> différences traditions diagrammes pour le monde occidental, arabe et byzantin
Pour la tradition grecque de l'\almageste, deux savants réalisent des commentaires de l'œuvre peu de temps après la disparition de Ptolémée.
Théon d'Alexandrie qui aurait vécu autour de 364 ap. J.-C. est l'un des commentateurs les plus prolifiques des travaux de Ptolémée. Cependant, son commentaire de l'\almageste n'a pas entièrement survécu aux différentes époques et certaines parties sont aujourd'hui manquantes\footcite{tihonLivreRetrouveCommentaire1987}. Vers 340 ap. J.-C., un deuxième savant grec, le mathématicien Pappus d'Alexandrie rédige un commentaire de l'\textit{Almageste} dans le \textit{Livre IV} de ses \textit{Collections mathématiques}. Il s'agit d'un ouvrage dans lequel il expose de manière complète et systématique toutes les connaissances de son époque en apportant des explications et des approfondissements\footcite{meyerPAPPUS1999}. 

Par la suite, l'œuvre se diffuse hors du monde hellénistique vers le monde arabe. Le savant persan Mohammad Nasir al-Din al-Tūs rédige un \textit{Tahrir al-Majiṣtī}, ce qui signifie \textit{Commentaire sur l'Almageste}\footcite{universalisMOHAMMADNASIRALDIN2008}. 


Averroès est aussi à l'origine d'un \textit{Abrégé de l'Almageste} écrit en 1159-1162. La particularité de cette œuvre vient du fait qu'elle ait été écrite en Andalousie au XIIe siècle dans un contexte de remise en question de l'astronomie ptolémaïque. En effet, des savants comme Maïmonide, Ibn Bajja ou Ibn Tufayl pensaient qu'il était nécessaire de réformer les idées de Ptolémée pour qu'elles concordent parfaitement avec la vision aristotélicienne du monde. Averroès fait parti de ce mouvement mais lorsqu'il rédige l'\textit{Abrégé de l'Almageste}, il n'a pas la volonté de corriger l'œuvre. En effet, il préfère attendre une réforme totale de l'astronomie ptolémaïque et se résigne donc à suivre les idées de son temps sur l'\textit{Almageste}.
Quelques petites erreurs involontaires se sont sans doute glissées dans son \textit{Abrégé de l'Almageste}. Même s'il possède un solide bagage scientifique, Averroès n'est pas astronome à l'origine\footcite{layAverroesAbregeDastronomie1998}. Nous ne possédons qu'une version de cette œuvre traduite en hébreu par Jacob Anatoli dans le cadre du mécénat de l'empereur Frédéric II au XIIIe siècle. Il n'existe plus aucun témoin de la version arabe et il semblerait que cette dernière n'ait jamais été traduite en latin\footcite{layAverroesHebraicusInedit2005}.

Plusieurs siècles plus tard, Gérard de Crémone importe l'œuvre de Ptolémée dans l'Occident chrétien. 
Ce dernier se rend à Tolède dans le but d'avoir accès aux traductions arabes de l'\textit{Almageste} alors que l'œuvre n'existe pas encore en latin. Grâce à sa connaissance de l'arabe et à ses solides connaissances en logique, mathématique et astronomie, Gérard de Crémone réalise la première traduction latine de l' \textit{Almageste}. Par la suite, il n'hésite pas à effectuer plusieurs révisions de la traduction en consultant de nouveaux manuscrits en arabe. Il existe donc plusieurs états du texte traduit par Gérard de Crémone. Paul Kunitzsch, un chercheur allemand en études arabes, a relevé trois états de sa traduction. Il a examiné trente-quatres témoins de l'œuvre mais seulement quatre d'entre-elles possède explicitement le nom de Gérard de Crémone. Il semblerait que des copies de son travail soient sorties progressivement de Tolède au fur et à mesure qu'il traduisait l'œuvre et qu'il apportait des corrections\footcite{jacquartTraductionsAuFil2018}.

La \gls{bnf} possède plusieurs copies de la traduction de l'\textit{Almageste} par Gérard de Crémone. Elle possède notamment la plus ancienne copie de son premier état, le manuscrit \textit{Latin 14738}\footcite{jacquartTraductionsAuFil2018}. Elle détient également une copie latine réalisée en 1213 à Paris avec des annotations\footcite{ptolemaeusPtolomeusAlmagestumTransl1213}, le manuscrit \textit{Latin 16200}. 
Dans l'une des initiales historiées du manuscrit, nous pouvons observer une illustration en hommage  à Ptolémée le représentant sous la forme d'un roi trônant en majesté\footcite{TraductionLatineLAlmageste}. 

Ces différentes copies, traductions et commentaires montrent à quel point la diffusion de l'œuvre est étendue. La vision géocentrique de l'univers de Ptolémée n'est remise en question qu'au XVIe siècle suite aux travaux de Nicolas Copernic. Dans son \textit{De Revolutionibus orbium caelestium}, il défend la théorie de l'héliocentrisme dans laquelle le Soleil se trouve au centre du monde\footcite{verdetHELIOCENTRISME2008}. \\



	\section[L'iconographie témoin des diffusions intellectuelles]{L'iconographie comme témoin des diffusions intellectuelles}

	Nous retrouvons au sein de ces sources scientifiques, un objet iconographique récurrent~: le diagramme. L'étude de ce dernier est un excellent moyen de retracer la diffusion d'une œuvre.
	Les diagrammes astronomiques sont des représentations iconographiques utilisées en astronomie pour illustrer les positions, les mouvements et les relations entre les corps célestes. \som{On a aussi des diagrammes mathématiques dans le corpus} Ces derniers ont joué un rôle crucial dans la diffusion du savoir astronomique depuis l'Antiquité. \som{Ce point justement devrait être largement abordé dans les parties précédentes, c'est l'élément intéressant du cadre historique.}

\subsection{Les différentes familles de diagrammes}
Nous pouvons regrouper les diagrammes en différentes familles. Néanmoins, les frontières de ces dernières ne sont pas totalement étanches. Il se peut que certains diagrammes aient des caractéristiques de plusieurs familles à la fois. Les diagrammes mathématiques servent à représenter des concepts géométriques ou arithmétiques par l'usage spécifique d'idiomes, de lignes, de cercles et d'étiquettes alphabétiques. Ils servent par exemple à calculer la position d'une planète ou d'une étoile. Les diagrammes astrologiques montrent la position des astres à un moment donné pour en tirer des prédictions ou des interprétations sur les activités humaines. Ils sont dessinés à l'aide de lignes, de cercles et d'un langage graphique. Les diagrammes d'instrument permettent de comprendre le fonctionnement ou la structure d'un instrument astronomique comme l'astrolabe ou le quadrant afin de réaliser une observation ou un calcul. Ils sont représentés à l'aide de graduations, de nombres, de lignes et de cercles plus métriques que dans les autres familles. Pour finir, le diagramme cosmologique montre l'organisation de l'univers dans son ensemble par le biais de l'utilisation d'un langage de couleur et de texture\footcite{Conference2025Long2025}.

\som{Dans la mesure où nous ne différencions pas les familles de diagramme dans le volet numérique du projet, est-ce bien utile ? Toujours se demander comment les sous-parties répondent à la problématique ou formulent un élément essentiel pour construire la réponse.}


\subsection{Construire le stemma codicum de l'œuvre à partir des diagrammes}

En 2014, Dominique Raynaud publie un article nommé \textit{"Building the stemma codicum from geometric diagrams. A treatise on optics by Ibn al-Haytham as a test case"} dans lequel il expose une idée novatrice : il veut établir le \textit{stemma codicum} d'une tradition écrite uniquement à partir de diagrammes\footcite{raynaudBuildingStemmaCodicum2014}.

Un \textit{stemma codicum} est une représentation de toutes les étapes de la transmission d'une œuvre sous la forme d'un arbre inversé en établissant des relations entre les différents manuscrits. Le but du philologue est alors de reconstituer le texte le plus proche du manuscrit original perdu que l'on nomme \og archétype \fg. Lorsqu'un texte est copié à plusieurs reprises, il constitue une \og tradition littéraire \fg dont les exemplaires sont nommés \og témoins \fg dans le domaine de la philologie. Pour tenter de reconstituer le \textit{stemma codicum} il est donc nécessaire de relever les différentes variantes provenant des divers manuscrits\footcite{pouliquenUsingLatticesReconstructing}.

Pour établir un \textit{stemma codicum} à partir de diagrammes, Dominique Raynaud emprunte à la biologie le principe de cladistique. Il s'agit \og d'une méthode de classification biologique qui exprime la phylogénie, c'est à dire les relations de parenté existant entre les êtres vivants \fg. Cette méthode \og repose sur le partage de caractères hérités d'une ascendance commune \fg, c'est-à-dire d'un \og ancêtre commun \fg \footcite{tassyCLADISTIQUE2012}.

Il explique que pour étudier les différents témoins d'une œuvre et les traditions qui l'entourent, il faut se baser sur les erreurs afin de réaliser un arbre généalogique. Nous pouvons affirmer qu'en utilisant cette démarche, il choisit de suivre l'approche de Lachmann. Il s'agit d'une méthode philologique élaborée au XIXe siècle par le philologue allemand Karl Lachmann dite de l'erreur commune. Ce dernier défendait la thèse suivante : Si un des témoins du texte présente une erreur, alors il y a de fortes chances que cette erreur soit aussi présente dans son descendant\footcite{pouliquenUsingLatticesReconstructing}.

Dominique Raynaud teste sa méthode sur un traité d'optique d'Ibn al-Haytham. En suivant cette démarche, il est arrivé à constituer le premier \textit{stemma} de diagrammes jamais publié. Ce dernier a été réalisé à partir des cinq témoins contenant l'œuvre : 


\begin{figure}[h]
	\centering
	\begin{subfigure}{0.48\linewidth}
		\centering
		\includegraphics[width=\linewidth]{images/diagram_stemma.png}
	\end{subfigure}
	\hfill
	\begin{subfigure}{0.48\linewidth}
		\centering
		\includegraphics[width=\linewidth]{images/text_stemma.png}
	\end{subfigure}
	\caption{Stemmata pour les diagrammes (à gauche) et pour le texte (à droite) \footcite{raynaudBuildingStemmaCodicum2014}}
	\label{fig:stemma}
\end{figure}


En y regardant plus en détail, nous pouvons voir qu'il y a une forte similitude entre le \textit{stemma} textuel et le \textit{stemma} des diagrammes. Nous pouvons en venir à la conclusion suivante : Quand les diagrammes sont intégrés au texte, il est envisageable de ne faire qu'un seul arbre pour les deux. Néanmoins, si les diagrammes ont été copiés a posteriori du texte ou qu'ils ont été corrigés par la suite, mieux vaut réaliser deux \textit{stemmata} distincts et étudier leur transmissions séparément. Le fait qu'ils aient été copiés par des personnes différentes à des moments distincts impacte fortement leur étude. Il est même plus avantageux de réaliser le \textit{stemma codicum} d'une tradition mathématique à partir des diagrammes. La densité d'erreur serait sept à huit fois plus élevée dans un diagramme géométrique que dans un texte occupant la même superficie. Dans la mesure où sa méthode repose sur le relevé d'erreurs, il est plus judicieux de réaliser un \textit{stemma} avec des diagrammes\footcite{raynaudBuildingStemmaCodicum2014}.\\

Comme nous venons de le voir, il est possible de retracer l'historique d'une œuvre grâce à ses diagrammes. Cependant, ces derniers sont des éléments iconographiques complexes. Il convient donc de les étudier de manière rigoureuse.

\som{Comme pour les parties sur la diffusion, je trouve que la sous-partie fait un peu trop "synthèse d'article" plutôt que sous-partie qui répond plus précisément à la problématique ; on a du mal à voir le rapport avec la partie précédente, il faudrait essayer de lien tout ça de manière plus fluide.s}

	\section{Limites des approches philologiques traditionnelles}
	\subsection{Altérations, pertes et déplacements dans la tradition manuscrite}


Pour étudier la diffusion d'une œuvre à travers ses diagrammes, il est important de prêter attention à la qualité de l'image, du texte qui l'accompagne et de l'aspect général de son manuscrit. En effet, cette dernière peut subir des changements significatifs au fil du temps dus à la conservation du manuscrit.

Dans les manuscrits, la qualité de représentation du diagrammes peut fortement varier. Certains d'entre eux sont tracés proprement à l'encre tandis que d'autres sont représentés de manière plus imparfaite à la mine de plomb par exemple. Le medium utilisé pour dessiner le diagramme a son importance car certains se dégradent plus facilement dans le temps, ce qui impacte sa lisibilité et son identification. 

L'état du parchemin peut aussi nuire à la clarté du diagramme. Ken Saito évoque dans son article sur l'évolution des éditions des \textit{Elements} d'Euclide au fil du temps qu'un manuscrit qu'il désigne sous le nom de \og manuscrit F \fg seraient l'une des meilleures lectures d'une partie de l'œuvre. Malheureusement elle est très endommagée ce qui complique sa lecture\footcite{saitoTraditionsDiagramTradition2012}. La dégradation du support rend la lecture compliquée et peut impacter son utilisation dans une étude philologique.

La pire situation qu'il puisse arriver à un ouvrage est évidemment sa perte qui rend son exploitation impossible.

En ce qui concerne l'organisation du manuscrit, les feuillets et les cahiers sont parfois déplacés, voire perdus. Cela complique la comparaison car l'ordre des images se retrouve aussi perturbée. 

Pour toutes ces raisons, l'historien des sciences peut être amenés à devoir accéder à comparer différents témoins. Cependant, parfois ces derniers peuvent être répartis dans différentes institutions dans le monde entier. La numérisation est une pratique essentielle au monde de la recherche. Cependant, en ce qui concerne les diagrammes, leur recherche et leur comparaison peut se révéler chronophage.




\subsection{Les différentes conventions de représentation d'un diagramme}

Le travail de l'historien des sciences peut s'avérer compliqué lorsqu'un concept ou une démonstration est représentée de manière différente dans différents témoins. En effet, les conventions et les choix graphiques des diagrammes diffèrent en fonction de l'époque, de l'endroit et du sujet d'étude.

Michela Malpangotto étudie ce phénomène en s'appuyant sur l'exemple de l'œuvre de Théodose les \textit{Sphériques} écrite au Ier siècle avant notre ère. Il s'agit d'un texte fondamental dans l'étude de la géométrie sphérique qui est structuré en cinquante-neuf propositions et divisé en trois livres. Il fait partie de ce que l'on nomme la \og Petite astronomie \fg, un recueil d'ouvrages compilés par les Grecs afin de faciliter la compréhension de l'\textit{Almageste} de Ptolémée. Il a été étudié et transmis pendant près de dix siècles. La géométrie sphérique est définie de la manière suivante : \og La géométrie sphérique étudie la sphère comme un objet solide mais surtout comme contexte spatial des éléments qui interagissent sur elle dans un agencement tridimensionnel complexe. \fg Il est alors nécessaire de mettre au même niveau, le plan du diagramme et l'agencement spatial des objets autour de la sphère. Cette dernière est un objet solide mais elle est surtout un contexte spatial pour des arcs, des segments de droites et des cercles qui y sont déterminés par l'intersection de différents plans inclinés dans l'agencement spatial tridimensionnel. Cependant, le concept de sphère n'est pas représenté de la même manière chez tous les auteurs \footcite{malpangottoGraphicalChoicesGeometrical2010}. 

Dans la version grecque originale, illustré ici par le manuscrit \textit{Vat. Gr. 204}, les deux parties de l'œuvre sont séparées par le choix de l'iconographie des diagrammes. Dans la première partie, nous retrouvons des diagrammes dans lesquelles la sphère n'est pas représentée. Il y a seulement des cercles produits par l'intersection du plan incliné de différentes façons qui sont représentés de manière juxtaposée dans le plan du diagrammes. Les arcs ainsi que les segments linaires sont aplatis et les objets placés de l'autre côté de la sphère sont retournés dans le plan de la figure. La conséquence majeure de ce mode de représentation est la dépendance des diagrammes vis-à-vis du texte. Il est nécessaire de lire les explications pour comprendre le diagramme. Dans la seconde partie de l'œuvre, les diagrammes sont construits en utilisant la perspective. Nous pouvons donc observer les éléments géométriques interagir entre eux à l'intérieur de cette dernière \footcite{malpangottoGraphicalChoicesGeometrical2010}.  

En Italie, Platon de Tivoli a réalisé trois éditions de cette œuvre au XVIe siècle en s'appuyant sur une version arabo-latine médiévale. Il choisit de représenter les diagrammes de manière schématique et plane comme dans la première partie de la version grecque originale de l'œuvre. L'édition de Francesco Maurolico marque un tournant dans la transmission des \textit{Sphériques}. En effet, ce dernier fait le choix de travailler sur la surface de la sphère qui, mise en avant, devient le contexte réel dans lequel les éléments géométriques interagissent. Christophe Clavius, un mathématicien allemand adopte cette iconographie dans son édition de 1586 qui sert de base à la tradition moderne des \textit{Sphériques}\footcite{malpangottoGraphicalChoicesGeometrical2010}. 

\begin{figure}[h]
	\centering
	\includegraphics[width=0.9\linewidth]{images/conventions_diagrammes.png}
	\caption{Différentes conventions de représentation de diagrammes évoquées par Michela Malpangotto\footcite{malpangottoGraphicalChoicesGeometrical2010}}
	\label{fig:conventions}
\end{figure}


L'étude de Michela Malpangotto nous montre l'existence de nombreuses conventions graphiques de diagrammes pouvant être très différentes en fonction des éditeurs et des époques. Lorsque nous tentons de retracer la diffusion d'une œuvre à partir de ces diagrammes, il est important de prendre en compte ce paramètre. Néanmoins, ces différentes manières de représenter les diagrammes sont aussi en elles-mêmes les témoins de l'évolution d'un concept et donc par extension des connaissances scientifiques. \\

La question de la représentation des diagrammes est aussi une problématique que nous retrouvons dans les éditions plus contemporaines. 

\subsection{La modification des diagrammes dans les éditions modernes des sources scientifiques}

Les modifications que s'autorisent à faire les historiens et éditeurs contemporains peuvent rendre la comparaison avec les sources anciennes compliquée. Dans l'article déjà cité précédemment, Ken Saito étudie les éditions modernes des diagrammes astronomiques des \textit{Elements} d'Euclide. Il expose la problématique suivante : les diagrammes que nous voyons dans les éditions imprimées à partir du XIXe siècle sont très différents de ceux présents dans les manuscrits médiévaux. Pourtant les témoins datant du Moyen Age sont les meilleures versions, voire les seuls exemplaires d'œuvres antiques mathématiques en absence de manuscrit datant de cette époque. La version qui sert de base à beaucoup d'éditions contemporaines est celle de Heiberg datant de 1883-1888. Cependant, ce dernier s'est contenté de recopier les diagrammes de Ferdinand August simplifiés dans un but pédagogique dans son édition datant des années 1820 \footcite{saitoTraditionsDiagramTradition2012}. Se pose alors la question des conventions d'édition. Deux points de vue s'opposent. Nous avons d'abord celui de la maison d'édition \textit{Les Belles Lettres} décrit dans leur \textit{Règles et recommandations pour les éditions critiques} qui explique qu'il est nécessaire de reproduire les diagrammes aussi précisément que possible sans essayer de les corriger ou de les modifier. Michael Hunter, lui, défend plutôt l'idée selon laquelle il est acceptable que les diagrammes soient redessinés pour que l'intention originelle de l'auteur soit transmise au lecteur\footcite{jardineCriticalEditingEarlyModern2010}.\\

Ces différentes manières de représenter un même diagramme peuvent poser problème aux chercheurs lorsque ces derniers cherchent à établir des liens entre les différents témoins d'une même œuvre ou tradition. Même pour un spécialiste qui connaît parfaitement les différentes conventions, identifier et comparer chaque diagrammes dans plusieurs témoins peut s'avérer très chronophage.

	\clearemptydoublepage

	\chapter{AIKON et l'automatisation des traitements}
	\section{Le choix de la vision artificielle pour ces objets d'étude~: la plateforme AIKON}

	\section{Extraction de regions / Calcul de similarités}

	\section{De l'annotation manuelle au traitement de masse~: les interfaces existantes}

	\clearemptydoublepage

	\chapter{Vers la nécessité d'interfaces d'exploration}
	\section{Masse de données et limites de l'exploration manuelle}

	\section{De la correction d'annotations à l'interprétation scientifique}

	\section{Questions de recherche et besoins d'exploration du corpus}

	\clearemptydoublepage

	\part{Méthodologie de conception des visualisations exploratoires}
	\chapter{La visualisation en histoire}
	\section{Evolution des représentations visuelles en histoire}

	\section{Visualisations de réseaux et étude des disséminations}

	\section{Enjeux critiques de la médiation numérique des données historiques}

	\clearemptydoublepage

	\chapter{Formalisation des besoins et contraintes de conception}
	\section{Analyse des pratiques et questions des chercheurs EIDA}

	\section{Cahier des charges~: objectifs scientifiques et contraintes techniques}

	\section{Choix méthodologiques~: network graph et alignement des sources (bipartite)}

	\clearemptydoublepage

	\chapter{Processus itératif de développement}
	\section{Prototypage et tests de faisabilité technique}

	\section{Mise en oeuvre concrète~: choix techniques et difficultés rencontrées}

	\section{Intégration des retours utilisateurs et ajustements}

	\clearemptydoublepage

	\part{Évaluation critique et perspectives d'intégration}
	\chapter{Analyse des visualisations produites}
	\section{Visualisation bipartite~: révéler les réorganisations du contenu intellectuel}

	\section{Network graph~: explorer les chaînes de transmissions}

	\section{Adéquation aux objectifs d'exploration scientifique}

	\clearemptydoublepage

	\chapter{Validation et limites méthodologiques}
	\section{Critères d'évaluation de l'efficacité exploratoire}

	\section{Retour d'expérience par les chercheurs}

	\section{Limites liées aux données et biais d'interprétation}

	\clearemptydoublepage

	\chapter{Perspectives d'évolution et généralisation}
	\section{Intégration dynamique à la plateforme AIKON}

	\section{Transférabilité vers d'autres corpus et projets}

	\section{Implications pour l'évolution des pratiques en humanités numériques}

	\clearemptydoublepage

	\chapterNo{Conclusion}
	\addcontentsline{toc}{chapter}{Conclusion}

	\appendix
	\part*{Annexes}
	\addcontentsline{toc}{part}{Annexes}

	\clearemptydoublepage

	\backmatter
	\printacronyms[title=Liste des acronymes,toctitle=Acronymes]
	\printglossary
	\printbibliography
	\tableofcontents

\end{document}